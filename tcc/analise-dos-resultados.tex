% ---
% ANÁLISE DOS RESULTADOS
%
% ---

\chapter{Análise dos Resultados}

O \textit{Java Embedded} é composto de diversos módulos para o desenvolvimento de \textit{software} embarcado, compatível com as necessidades encontradas no conceito de\textit{ Internet} das Coisas.

No \textit{Java ME Embedded} são encontradas diversas ferramentas para o desenvolvimento, juntamente com uma rica documentação do \textit{Device I/O API}. Entretanto a plataforma não é fixa em apenas um sistema operacional, pois existem recursos voltados para \textit{Windows} e outros para \textit{Linux}. A parte embarcada é disponibilizada por uma estrutura de projeto com diversos \textit{scripts} para a execução e/ou interfaceamento com o sistema operacional da aplicação desenvolvida.

No \textit{Java Embedded Suite} são disponibilizadas os principais módulos para aplicações com servidores através de versões reduzidas e otimizadas para o ambiente embarcado, no conceito \textit{Internet} das Coisas faz uso de tais aplicações por conectividade com outros sistemas e usuários. A plataforma possui interoperabilidade com demais plataformas do \textit{Java Embedded}, trazendo um grande escopo de fonte de dados. O desenvolvimento é compatível com o tradicional no Java EE (Edição Empresarial - \textit{Enterprise Edition}), utilizando um número menor de componentes e sua implantação através de \textit{scripts}, como no \textit{Java ME Embedded} com a vantagem do interfaceamento com o sistema operacional.