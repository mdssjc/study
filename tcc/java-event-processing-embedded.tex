% ---
% JAVA ORACLE EVENT PROCESSING
%
% ---

\chapter{Java Oracle Event Processing}

Este capítulo tem como objetivo demonstrar a plataforma \textit{Java Embedded} 
- \textit{Java Oracle Event Processing} (\textit{JOEP}).

\section{Instalação}

A plataforma de prototipagem deve estar operacional com o sistema operacional 
\textit{Linux} e a plataforma \textit{Java 8}, no caso de estudo a plataforma 
\textit{Raspberry PI B+}. A figura \ref{fig:joep/configuracao} apresenta as 
configurações do sistema em estudo.

\begin{figure}[H]
    \centering
    \includegraphics[width=0.7\linewidth]{figuras/java/configuracao}
    \caption{Configuração do Sistema}
    \label{fig:joep/configuracao}
\end{figure}

Para a plataforma de prototipagem é necessário a instalação do pacote 
\textit{Oracle Java Embedded Suite 7.0} e \textit{Oracle Event Processing for 
Oracle Java Embedded}, necessário ter uma conta no site da 
\textit{Oracle} para realizar o \textit{download}.

Transfira os arquivos \newline
\verb|jes-7.0-ga-bin-b11-linux-arm-runtime-15_nov_2012.zip|
, \newline
\verb|jes-7.0-ga-b11-linux-samples-15_nov_2012.zip|
e \newline
\verb|ofm_oep_embedded_11_1_1_7_1_linux.zip|
 do sistema operacional principal para o dispositivo remoto, basta utilizar o 
comando \textit{SCP}:

\verb|scp </path/from/file> <user>@<address>:/path/to/destination|

Descompacte os arquivos, basta utilizar o comando \textit{UNZIP}:

\verb|unzip <file>|

\section{Aplicações para a Internet das Coisas}

O \textit{Java Oracle Event Processing} permite o processamento em tempo real 
dos dados amostrados através dos periféricos e recursos encontrados nos 
dispositivos, assim aplicando técnicas de filtragens e estatísticas para a 
realização de análises.

\section{Resultados}

Os resultados demonstram a visão na qual a plataforma \textit{Java Oracle Event 
Processing} contribui para o desenvolvimento de aplicações destinadas a 
\textit{Internet} das Coisas.

\subsection{Positivos}

\begin{itemize}
    
    \item Grande poder de processamento, por oferecer uma gama de recursos para 
    trabalhar com dados;
    
    \item Utilização de linguagem de consulta para a manipulação e amostragem 
    dos dados;
    
    \item Traz a capacidade de \textit{gateway} para o dispositivo, quando 
    conectado em outros dispositivos.
    
\end{itemize}

\subsection{Negativos}

\begin{itemize}
    
    \item Não possui uma interface de desenvolvimento produtiva, como 
    encontrada na versão tradicional.
    
\end{itemize}
